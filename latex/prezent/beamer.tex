%%%%%%%%%%%%%%%%%%%%%%%%%%%%%%%%%%%%%%%%%%%%%%%%%%%%%%%%%%%%%%%%%%%%%%%%%%%%%%
% Plik ten należy, co najmniej dwukrotnie, przetworzyć przy użyciu komendy 'pdflatex'
%
% Stanisław Polak, 25-01-2017
%%%%%%%%%%%%%%%%%%%%%%%%%%%%%%%%%%%%%%%%%%%%%%%%%%%%%%%%%%%%%%%%%%%%%%%%%%
%%%%%%%%%%%%%%%%%%%%%%%%%%%%%%%%%%%% Włączenie, domyślnego, trybu "beamer" 
%%%%%%%%%%%%%%%%%%%%%%%%%%%%%%%%%%%%%%%%%%%%%%%%%%%%%%%%%%%%%%%%%%%%%%%%%%
\documentclass[polish,xcolor=table,9pt,aspectratio=1610,hyperref={pdfpagemode=FullScreen}]{beamer} 
% Opcje:
% 	[polish]                            włączenie polonizacji
% 	[xcolor=table]                      włączenie możliwości kolorowania tabel - ładowanie pakietu ,,colortbl''
% 	[ascpectratio=1610]                 slajdy w proporcji 16:10, domyślnie 4:3
% 	[9pt]                              	podstawowy font ma mieć rozmiar '9pt', domyślna wartość to '11pt'. Dostępne wartości: 8pt, 9pt, 10pt, 11pt, 12pt, 14pt, 17pt, 20pt
% 	[hyperref={pdfpagemode=FullScreen}]	przeglądarka PDF ma automatycznie wyświetlać prezentację w trybie 'pełny ekran'

%%%%%%%%%%%%%%%%%%%%%%%%%%%%%%%%%%%%%%%%%%%%%%%%%%%%%%%%%%%%%%%%%%%%%%%%%%%%%%%%%%%%%% 
% Jeżeli chcesz otrzymać prezentacje w postaci materiałów drukowanych (dla słuchaczy),
% to należy zakomentować linię 8:  '\documentclass[...]{beamer}', 
% a odkomentować linię  23:'\documentclass[handout,...]{beamer}'
%%%%%%%%%%%%%%%%%%%%%%%%%%%%%%%%%%%%%%%%%%%%%%%%%%%%%%%%%%%%%%%%%%%%%%%%%%%%%%%%%%%%%% 
%%%%%%%%%%%%%%%%%%%%%%%%%%%%%%%%%%%% Włączenie trybu "handout" %%%%%%%%%%%%%%%%%%%%%%%
%%%%%%%%%%%%%%%%%%%%%%%%%%%%%%%%%%%%%%%%%%%%%%%%%%%%%%%%%%%%%%%%%%%%%%%%%%%%%%%%%%%%%% 
%\documentclass[handout,polish,xcolor=table,9pt,aspectratio=1610]{beamer} 

%%%%%%%%%%%%%%%%%%%%%%%%%%%%%%%%%%%%%%%%%%%%%%%%%%%%%%%%%%%%%%%%%%%%%%%%%%%%%%%
\usepackage[utf8]{inputenc}
\usepackage{polski}   %włączenie obsługi polskich liter
\usepackage{babel}    %aby zadziałała polonizacja 'beamera'
\usepackage{graphicx} %włączenie obsługi grafiki
\DeclareGraphicsRule{.pdftex}{pdf}{*}{}
\AtBeginPart{\frame{\partpage}} %Na początku każdej z części prezentacji, wyświetlaj jej tytuł
%%%%%%%%%%%%%%%%%%%%%%%%%%%%%%%%%%%%%%%%%%%%%%%%%%%%%%
% Komendy, które mają być wykonywane w trybie 'beamer'
%%%%%%%%%%%%%%%%%%%%%%%%%%%%%%%%%%%%%%%%%%%%%%%%%%%%%%
\mode<beamer>{ 
  %%%%%%%%%%%%%%%%%%%%%%%%%%%%%%%%%%%%%%%%%%%%%%%%%%%%%%%%%%%%%%%%%%%%%%%%%%%%%%%
  %'CambridgeUS' jest przykładowym stylem (wystrojem) - inne style można obejrzeć na stronie 
  % http://gknor.keep.pl/beamer/beamer_motywy.html
  % Zastosowany wystrój ma wpływ na wygląd struktury slajdu (nagłówki, stopki, rodzaje obramowania, itp.) jak i jego kolorystyki
  %%%%%%%%%%%%%%%%%%%%%%%%%%%%%%%%%%%%%%%%%%%%%%%%%%%%%%%%%%%%%%%%%%%%%%%%%%%%%%%
  % Na stronie https://github.com/polaksta/beamer-AGH można znaleźć styl AGH
  %%%%%%%%%%%%%%%%%%%%%%%%%%%%%%%%%%%%%%%%%%%%%%%%%%%%%%%%%%%%%%%%%%%%%%%%%%%%%%%
	\usetheme{CambridgeUS} %Określanie stylu (wystroju) prezentacji
	\usecolortheme{orchid} %Określanie schematu kolorów - zmiana kolorystyki slajdu
}
%%%%%%%%%%%%%%%%%%%%%%%%%%%%%%%%%%%%%%%%%%%%%%%%%%%%%%%
% Komendy, które mają być wykonywane w trybie 'handout'
%%%%%%%%%%%%%%%%%%%%%%%%%%%%%%%%%%%%%%%%%%%%%%%%%%%%%%%
\mode<handout>{
  \usepackage{pgfpages}
  \pgfpagesuselayout{4 on 1}[a4paper,border shrink=5mm,landscape] %Wynikowy dokument PDF będzie zawierał cztery slajdy na jednej stronie kartki formatu A4
  \usetheme{boxes} %Określanie stylu (motywu) prezentacji
  \addheadbox{structure}{\quad\insertsubsection\hfill\insertsection\hfill\inserttitle\qquad} %Określanie zawartości nagłówka kartki
  \addfootbox{structure}{\quad\insertauthor\hfill\insertframenumber\hfill\insertsubtitle\qquad} %Określanie zawartości stopki kartki
}
%%%%%%%%%%%%%%%%%%%%%%%%%%%%%%%%%%%%%%%%%%%%%%%
% Określamy tytuł, autora, afiliację  oraz datę prezentacji
% UWAGA, poniższe komendy tylko definiują treść slajdu tytułowego
% Jeżeli chcemy umieścić te informacje na slajdzie należy
% w ciele dokumentu umieścić komendę '\maketitle' lub '\frame{\titlepage}'
%%%%%%%%%%%%%%%%%%%%%%%%%%%%%%%%%%%%%%%%%%%%%%%
\title{Framework e-commerce}
\subtitle{Elastyczny szablon sklepu internetowego}
\author{Przemysław Magiera}
\begin{document}

%%%%%%%%%%%%%%%%%%%%%%%%%%%%%%%%%%%%%%%%%
%%%%% Slajd ze stroną tytułową %%%%%%%%%%
%%%%%%%%%%%%%%%%%%%%%%%%%%%%%%%%%%%%%%%%%
% Komenda \tilepage tworzy stronę tytułową na podstawie \title, \author, \institute oraz \date zawartych w preambule (patrz wyżej)
\frame{\titlepage}

%%%%%%%%%%%%%%%%%%%%%%%%%%%%%%%%%%%%%%%%%%%
%%%%% Slajd z planem prezentacji %%%%%%%%%%
%%%%%%%%%%%%%%%%%%%%%%%%%%%%%%%%%%%%%%%%%%%i
%Prezentację można, opcjonalnie, podzielić na części
%Przy pomocy rozkazu \part rozpoczynamy nową część. 
%Poszczególne części są od siebie niezależne
%%%%%%%%%%%%%%%%%%%%%%%%%%%%%%%%%%%%%%%%%%%i
\part{Zdefiniowane problemy} %Tytuł części
%%%%%%%%%%%%%%%%%%%%%%%%%%%%%%%%%%%%%%%%%%%i
\begin{frame}{Plan prezentacji}
% Komenda \tableofcontents tworzy, w obrebie danej części, plan prezentacji, na podstawie treści rozkazów
% tworzenia rozdziałów i podrozdziałów czyli rozkazów \section oraz \subsection.
% Jeśli w prezentacji nie używamy ani \section ani  \subsection, to  plan prezentacji należy utworzyć ręcznie 
\tableofcontents
\end{frame}

%%%%%%%%%%%%%%%%%%%%%%%%%%%%%%%%
%%%%% Tytuł rozdziału %%%%%%%%%%
%%%%%%%%%%%%%%%%%%%%%%%%%%%%%%%%
\section{Zdefiniowane problemy}
% Jeśli tytuł rozdziału jest długi i jego nazwa nie będzie się się mieścić 
% w panelu górnym slajdu, należy użyć tego rozkazu w następującej postaci:
%
% \section[krótka nazwa rozdziału]{długa nazwa rozdziału}

%%%%%%%%%%%%%%%%%%%%%%%%%%%%%%%%%%%
%%%%% Tytuł podrozdziału %%%%%%%%%%
%%%%%%%%%%%%%%%%%%%%%%%%%%%%%%%%%%%
\subsection{Problemy - zagadnienie 1}

%%%%%%%%%%%%%%%%%%%%%%%
%%%%% Slajd  %%%%%%%%%%
%%%%%%%%%%%%%%%%%%%%%%%
\begin{frame}{Architektura sklepów internetowych}{Co można powiedzieć o archotekturze i rozszerzalności sklepów internetowych?}
\begin{itemize}
\item<1-> sklepwy nieoparte o frameworki są trudne w utrzymaniu %Umieść ten podpunkt w warstwie 1 i kolejnych
\item<2-> dodawanie nowych funkcjonalności jest bardzo kosztowne  %Umieść ten podpunkt w warstwie 2
\item<2-> każda nowa funkcjonalność wymaga implementacji interfejsu do administracji i zarządzania
\end{itemize}
\end{frame}

\subsection{Problemy - zagadnienie 2}

%%%%%%%%%%%%%%%%%%%%%%%
%%%%% Slajd  %%%%%%%%%%
%%%%%%%%%%%%%%%%%%%%%%%
\begin{frame}{Wydajność i skalowalność}{Jaka jest wydajność sklepów internetowych?}
	\begin{itemize}
		\item<1-> często klasyczne sklepy wymagają skalowania pionowego, które jest bardzo drogie
		\item<2-> frameworki nie oferują mechanizmów szybkiego dostępu do najbardziej kluczowych danych - katalog produktowy %Umieść ten podpunkt w warstwie 2
		\item<3-> najbardziej obciążone punkty aplikacji nie są odseparowane od reszty 
	\end{itemize}
\end{frame}

\subsection{Problemy - zagadnienie 3}

%%%%%%%%%%%%%%%%%%%%%%%
%%%%% Slajd  %%%%%%%%%%
%%%%%%%%%%%%%%%%%%%%%%%
\begin{frame}{Katalog produktowy}{Czy katalogi produktowe są zawsze proste i spełniają swoje zadanie?}
	\begin{itemize}
		\item<1-> główne zadanie katalogu produktowego to zapewnienie łatwego i szybkiego sposobu na dotarcie do interesującej informacji
		\item<2-> brak możliwości konfiguracji i modyfikacji wyszukiwarki %Umieść ten podpunkt w warstwie 2
		\item<3-> trudność w dotarciu do informacji
	\end{itemize}
\end{frame}

\subsection{Problemy - zagadnienie 4}

%%%%%%%%%%%%%%%%%%%%%%%
%%%%% Slajd  %%%%%%%%%%
%%%%%%%%%%%%%%%%%%%%%%%
\begin{frame}{Stosowane technologie}{Czy frameworki są na tyle elastyczne aby nie zostać \textit{w tyle}?}
	\begin{itemize}
		\item<1-> stosowanie zamkniętych komercyjnych technologii 
		\item<2-> programiści nie widzą źródeł i nie mają wpływu na rdzeniowe elementy platformy, na której programują 
		\item<3-> w przypadku błędów, nadpisanie komponentow platformy jest bardzo trudne lub nawet niemożliwe
	\end{itemize}
\end{frame}

\part{Zastosowane rozwiązania problemów}

\section{Zastosowane rozwiązania problemów}

%%%%%%%%%%%%%%%%%%%%%%%
%%%%% Slajd  %%%%%%%%%%
%%%%%%%%%%%%%%%%%%%%%%%
\subsection{Dynamiczny panel administracyjny}

\begin{frame}{Dynamiczny panel administracyjny}{Zarządzanie funkcjonalnościami out-of-the-box}
	\begin{itemize}
		\item<1-> generowana tabelka dla każdej encji danego rodzaju np. kategorii 
		\item<2-> generowany formularz edycji dowolnej encji
		\item<3-> generowany mechanizm zarządzania relacjami dowolnej klasy np. dzieci kategorii
	\end{itemize}
	
\end{frame}

\begin{frame}{Zarządzanie nadpisanymi klasami przez panel}{Zarządzanie funkcjonalnościami, które mogą zostać dopisane do platformy}
	\begin{itemize}
		\begin{exampleblock}{Przykład}
			Programista decyduje się na 
		\end{exampleblock}
		\item<1-> platforma będzie świadoma tego, że 
		\item<2-> generowany formularz edycji dowolnej encji
		\item<3-> generowany mechanizm zarządzania relacjami dowolnej klasy np. dzieci kategorii
	\end{itemize}
\end{frame}
%%%%%%%%%%%%%%%%%%%%%%%
%%%%% Slajd  %%%%%%%%%%
%%%%%%%%%%%%%%%%%%%%%%%
\begin{frame}{Dynamiczny panel administracyjny}{Panel administracyjny, który umożliwi zarządzanie wszystkimi encjami w systemie}
\begin{exampleblock}{Przykład}
przykład 1
\end{exampleblock}
%%%%%%%%%%%%%%%%%%%
\begin{definition}
Dystrybucją \LaTeX a nazywamy \ldots
\end{definition}
%%%%%%%%%%%%%%%%%%%
\begin{theorem}[Pitagorasa]
$a^2+b^2=c^2$
\end{theorem}
%%%%%%%%%%%%%%%%%%%
\pause %To, co poniżej ma być umieszczone w kolejnej warstwie oraz wszystkich następnych
\begin{proof}
Treść dowodu
\end{proof}
%Jeśli rysunek1.pdf jest wielostronicowym rysunkiem (np. utworzonym za pomocą programu 'Ipe'), parametr 'page=nr' powoduje wyświetlenie strony o podanym numerze
%\includegraphics<1>[page=1,scale=0.5,angle=-90]{rysunek1.pdf}
%\includegraphics<2>[page=2,scale=0.5,angle=-90]{rysunek1.pdf}
\end{frame}

%%%%%%%%%%%%%%%%%%%%%%%%%%%%%%%%%%%
%%%%% Tytuł podrozdziału %%%%%%%%%%
%%%%%%%%%%%%%%%%%%%%%%%%%%%%%%%%%%%
\subsection{Zagadnienie 1.2}

%%%%%%%%%%%%%%%%%%%%%%%
%%%%% Slajd  %%%%%%%%%%
%%%%%%%%%%%%%%%%%%%%%%%
\begin{frame}{Tytuł slajdu 3}
treść slajdu 3
%\scalebox{0.5}{ \input{rysunek2.pdftex_t} }  
\end{frame}

%%%%%%%%%%%%%%%%%%%%%%%
%%%%% Slajd  %%%%%%%%%%
%%%%%%%%%%%%%%%%%%%%%%%
\begin{frame}{Tekst w wielu kolumnach}
% Otoczenie 'columns' pozwala na składanie tekstu w wielu kolumnach.
% Poszczególne kolumny definiuje się za pomocą komendy 'column'
\begin{columns}
%Utwórz kolumnę o rozmiarze: 60% szerokości tekstu 
\column{0.6\textwidth}
\begin{description}
\item[Pojęcie 1] \pauza wyjaśnienie pojęcia 1
\item[Pojęcie 2] \pauza wyjaśnienie pojęcia 2
\end{description}

%Utwórz kolumnę o rozmiarze: 40% szerokości tekstu 
\column{0.4\textwidth}
\structure{Treść prawej kolumny}
\begin{enumerate}
\item pozycja 1
\item pozycja 2
\end{enumerate}
\end{columns}
\end{frame}

%%%%%%%%%%%%%%%%%%%%%%%%%%%%%%%%
%%%%% Tytuł rozdziału %%%%%%%%%%
%%%%%%%%%%%%%%%%%%%%%%%%%%%%%%%%
\section{Zagadnienie 2}

%%%%%%%%%%%%%%%%%%%%%%%
%%%%% Slajd  %%%%%%%%%%
%%%%%%%%%%%%%%%%%%%%%%%
\begin{frame}{Tytuł slajdu 5}
%Określamy kolor wierszy parzystych i nieparzystych tabeli w oparciu o aktualną kolorystykę bloku
\rowcolors{2}{block title.bg}{block body.bg}
\begin{table}
\begin{tabular}{|r|p{4cm}|l|}\hline
\textbf{Kol$_{1}$} &\textbf{Kol$_{2}$} &\textbf{Kol$_{3}$}\\
\hline
\hline
& & \\
& & \\
& & \\
& & \\
& & \\
& & \\
\hline
\end{tabular}
\end{table}
\end{frame}

%%%%%%%%%%%%%%%%%%%%%%%
%%%%% Slajd  %%%%%%%%%%
%%%%%%%%%%%%%%%%%%%%%%%
\part{Literatura}
% Aby tytuł rozdziału, tu  'Literatura', nie pojawił się na slajdzie z planem prezentacji, 
% należy użyć rozkazu \section* (gwiazdka po słowie 'section') 
\section*{Literatura}
\begin{frame}[allowframebreaks]{Literatura}
%Parametr 'allowframebreaks' oznacza, że jeżeli treść nie mieści się na jednym slajdzie, to LaTeX powinien utworzyć dodatkowe slajdy
\begin{thebibliography}{}
% Komenda  '\setbeamertemplate{bibliography item}[article]' oznacza, że poszczególne pozycje są artykułami i~w~związku z~tym każda z~pozycji będzie poprzedzona ikoną artykułu
% Domyślnie, poszczególne pozycje są poprzedzone ikoną artykułu, dlatego tego rozkazu nie musimy tu umieszczać
% \setbeamertemplate{bibliography item}[article]
    \bibitem{poz1}
      Imię i~nazwisko autora 
      \newblock
      Tytuł artykułu 
      \newblock 
      pozostałe informacje (nazwa czasopisma, numer, \ldots) 
      \newblock
      informacje dodatkowe (adres strony WWW, \ldots)
    \bibitem{poz2}
      Imię i~nazwisko autora 
      \newblock 
      Tytuł artykułu 
      \newblock
      pozostałe informacje (nazwa czasopisma, numer, \ldots)
      \newblock informacje dodatkowe (adres strony WWW, \ldots)
    
  %Komenda '\setbeamertemplate{bibliography item}[book] oznacza, że poszczególne pozycje są książkami i~w~związku z~tym każda z~pozycji będzie poprzedzona ikoną książki
  \setbeamertemplate{bibliography item}[book]
    \bibitem{poz3} 
      Imię i~nazwisko autora
      \newblock
      Tytuł książki
      \newblock
      pozostałe informacje (rok wydania, wydawca, \ldots)
      \newblock informacje dodatkowe (adres strony WWW, \ldots)
    \bibitem{poz4} 
      Imię i~nazwisko autora
      \newblock 
      Tytuł książki
      \newblock
      pozostałe informacje (rok wydania, wydawca, \ldots)
      \newblock informacje dodatkowe (adres strony WWW, \ldots)
    
  \setbeamertemplate{bibliography item}[online]
    \bibitem{poz5}
      Wujek Google
      \newblock
      \url{http://www.google.pl/}
\end{thebibliography}
%%%%%%%%%%%%%%%%%%%%%%%%%%%%%%%%%%%%%%%%%%%%%%%%%%%%%%%%%%%%%%%%%%%%%%%%%%
% Jeśli literaturę mamy w pliku 'literatura.bib' i chcemy używać Bibtex-a,
% należy odkomentować poniższe linie oraz usunąć całe 'thebibliography'
%%%%%%%%%%%%%%%%%%%%%%%%%%%%%%%%%%%%%%%%%%%%%%%%%%%%%%%%%%%%%%%%%%%%%%%%%%
%\bibliography{literatura}
%\bibliographystyle{plplain}
%\nocite{poz1}
%\nocite{poz2}
\end{frame}
\end{document}
