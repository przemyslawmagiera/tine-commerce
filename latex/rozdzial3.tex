\chapter{Implementacja systemu}
\thispagestyle{chapterBeginStyle}

\section{Opis technologii}
W projektcie użyto wielu technologi oraz frameworków. Wszystkie oparte są o język Java, dokumentację można znaleźć w \cite{Java-doc}. Interfejs użytkownika zaprojektowano przy użyciu szablonów opartych o projekt Start Bootstrap, którego opis można znaleźć na stronie \cite{sb}. 

\subsection{Spring Framework/Spring Boot}
Główną technologią jest framework Spring. Ułatwia pisanie aplikacji w Javie, w szczególności webowych, ze względu na dużą elastyczność i wzorzec projektowy Model-View-Controller, singleton oraz spełnia zasadę \textit{Inversion of Control}. Aplikacja jest oparta o platformę Spring Boot. Jest to nakładka na framework Spring, który przy obecnych standardach stał się skomplikowany w konfiguracji. Spring Boot zapewnia autokonfigurację wielu komponentów systemu, łatwe dodawanie modułów Springowych (np. security lub web). Co najważniejsze posiada wbudowany kontener na aplikację internetową, przez co nie ma potrzeby konfigurowania osobnego kontenera (np. Tomcat) i umieszczania tam aplikacji. Znacznie przyśpiesza to dewelopment. Więcej informacji najduje się w dokumentacji \cite{springb-docs} oraz w książce \cite{springbook}.

\subsection{Hibernate/JPA 2.1} 
Java Persistence API jest interfejsem służącym do komunikacji aplikacji z bazą danych. Jego implementacją jest framework Hibernate. Korzystanie z tych standardów ułatwia komunikację z bazą danych i zapewnia wiele przydatnych funkcjonalności. Między innymi zapewnia również obsługę transakcji. W połączeniu ze Spring Data to potężne narzędzie, a jednocześnie jest proste w użytku. Więcej informacji o frameworku Hibernate i JPA w książce \cite{JPA-hib}. 
\subsection{Spring Data}
Spring Data ułatwia konfigurację połączeń między bazą danych a serwerem. Do tego zapewnia automatyczne generowanie prostych kwerend bazodanowych, korzystając z zasady \textit{query by convention}\footnote{query by convention — sposób generowania kwerend bazodanowych na podstawie nazw metod w interfejsach}. Jest to sposób pisania nazw metod w interfejsach, które framework jest w stanie zaimplementować. Całość powoduje odejście tradycyjnej warstwy DataAccessObject. Framework pozwala na samodzielne implementowanie trudniejszych kwerend, w spersonalizowany sposób. 

\subsection{Apache Solr} 
Oparta na silniku Lucene i uruchomiona na osobnym serwerze wyszukiwarka zapewnia odseparowanie najbardziej obciążonej części sklepu. Dzięki Solrowi w implementowanym frameworku jest obecna bardzo szybka wyszukiwarka z możliwością wyszukiwania pełnotekstowego, filtrowania, sortowania i zawężania wyników względem różnych pól produktu. Książka \cite{solrbook} i dokumentacja \cite{Solr-doc} opisują działanie i możliwości Solra. Do integracji serwera Apache Solr z implementowanym frameworkiem użyto biblioteki SolrJ \cite{solrJ}.

\section{Omówienie kodów źródłowych}
Projekt składa się z 3986 linii kodu w języku Java, dlatego aby zachować zwięzłość i zrozumiałość w ramach omówienia kodów źródłowych zostanie przedstawiona tylko jedna funkcjonalność odnosząca się do konkretnych przypadków użycia zdefiniowanych w sekcji \textbf{Przypadki użycia} rozdziału \textbf{Projekt systemu}. Większość funkcjonalności jest napisana w analogiczny sposób ze względu na wzorzec architektoniczny model-view-controller, dlatego dokładna analiza jednej funkcjonalności wystarczy do zrozumienia metodologi wytwarzania kodu w projekcie. 

Na rysunku \ref{useCaseProgrammer} znajduje się use-case: \textit{Zarządzanie encjami}. Związany jest przypadek z diagramu \ref{dynEntFormUC} \textit{Wyświetlenie i manipulacja relacjami encji}.  Te dwa przypadki wiążą się z konstrukcją dynamicznego formularza edycji dla danej encji. Dla lepszego zrozumienia warto przypomnieć diagramy związane z implementacją tej funkcjonalności. Na rysunku \ref{konsFormEnc} przedstawiono diagram aktywności, zaś rysunek \ref{klasy_formularz_encyjny} przedstawia klasy użyte do implementacji. Dla przykładu działania kodu źródłowego zostanie przeanalizowana budowa formularza dla dowolnej encji.

\begin{small}
\begin{lstlisting}[language=Java, frame=lines, numberstyle=\tiny, stepnumber=5, caption=Przetworzenie zapytania żądania edycji encji \texttt{DynamicEntityController.java}\label{listing_dec}., firstnumber=1]
@GetMapping("/entities/{entityCode}/{id}/edit")
public String getDynamicEntityForm(final Model model, 
    @PathVariable(name = "entityCode") String entityCode,
    @PathVariable(name = "id") Long id) {
    model.addAttribute(ATTR_MENU_ITEMS , adminMenuGroupRepository.findAll());

    String className = resolveEntityCode(entityCode);
    if(permissionService.hasPermissionForOperation(className)) {
        DynamicForm dynamicForm = new DynamicForm();
        dynamicForm.setEntityClass(entityCode);
        List<RelationMetadata> relationalEntities = new ArrayList<>();

        dynamicEntityService.constructDynamicEntityForm(className, id,
                             dynamicForm, relationalEntities);

        model.addAttribute("relationalEntities", relationalEntities);
        model.addAttribute("dynamicForm", dynamicForm);
        model.addAttribute("entityName", entityCode);
        model.addAttribute("id", id);
    } else {
        model.addAttribute("authError", true);
    }
    return "dynamicEntityForm";
}
\end{lstlisting} 
\end{small}

Kod źródłowy \ref{listing_dec} przedstawia kontroler odpowiadający za przetworzenie żądania konstrukcji formularza edycji. Na przykład w przypadku produktu o \texttt{id = -105} mapowanie (url) będzie wyglądało następująco: \texttt{/entities/product/-105/edit}, gdzie \texttt{product} to kod encji \texttt{entityCode}. Na początku metody następuje dodanie do modelu menu, które jest widoczne na każdej stronie administracyjnej. Następnie w metodzie \texttt{resolveEntityCode}, kod encji produkt zostaje przetłumaczony na pełną nazwę klasy z której pochodzi. W tym celu system sprawdza, czy w bazie danych encja o takim kodzie należy do tych zdefiniowanych jako możliwe do obsłużenia w panelu administracyjnym. Kod źródłowy tej metody znajduje się na listingu \ref{listing_klasa}. 
\begin{small}
	\begin{lstlisting}[language=Java, frame=lines, numberstyle=\tiny, stepnumber=5, caption=Zamiana kodu encji na nazwę klasy: \texttt{DynamicEntityController.java}\label{listing_klasa}., firstnumber=1]
	private String resolveEntityCode(String entityCode){
	    return adminMenuItemRepository.findByCode(entityCode)
	           .map(AdminMenuItem::getClassName)
	           .orElse("");
	}
	\end{lstlisting} 
\end{small}

Mając nazwę klasy, sprawdzane jest, czy zalogowany użytkownik posiada prawa do edycji i wyświetlania danej klasy. Dzieje się to w metodzie \texttt{hasPermissionForOperation}. Zalogowany użytkownik pobierany jest z bazy danych, po czym jego wszystkie uprawnienia zostają zebrane w metodach \texttt{getAllChildrenPermissions} za pomocą rekurencyjnego przejścia po drzewie uprawnień, które należą do użytkownika administracyjnego. W tym przypadku każde uprawnienie, które jest mu przypisane, odgrywa rolę korzenia (przechodzimy po wielu drzewach). Każde uprawnienie będące potomkiem korzenia, jest brane pod uwagę.

\begin{small}
	\begin{lstlisting}[language=Java, frame=lines, numberstyle=\tiny, stepnumber=5, caption=Sprawdzenie uprawnień i algorytm przeszukiwania drzewa: \texttt{PermissionService.java}\label{listing_permission}., firstnumber=1]
	public boolean hasPermissionForOperation(String checkedClassName){
	    AdminUser adminUser = adminUserRepository
	                          .findByUsername(getLoggedInUsername()).get();
	    return getAllChildPermissions(adminUser.getAdminPermissions())
	           .stream()
	           .map(AdminPermission::getClassName)
	           .anyMatch(checkedClassName::equals);
	}
	
	private Set<AdminPermission> getAllChildPermissions(
	Set<AdminPermission> adminPermissions) {
	    final Set<AdminPermission> result = new HashSet<>();
	    adminPermissions.forEach(adminPermission -> 
	        getAllChildPermissions(adminPermission, result));
	    return result;
	}
	
	private void getAllChildPermissions(AdminPermission adminPermission, 
	                                    Set<AdminPermission> result) {
	    adminPermission.getChildPermissions().forEach(child ->     
	        this.getAllChildPermissions(child, result));
	    result.add(adminPermission);
	}
	\end{lstlisting} 
\end{small}

Jeżeli algorytm przeszukiwania drzewa znajdzie nazwę klasy w finalnej kolekcji uprawnień administratora, kontroler z listingu \ref{listing_dec} rozpoczyna budowę formularza. Formularz konstruowany jest w metodzie \texttt{constructDynamicEntityForm} przedstawionej na listingu \ref{listing_form}. Korzystając z metody \texttt{getAllFieldsFromClass} wyciąga z klasy (i wszystkich możliwych nadklas) wszystkie pola, które są objęte interfejsem \texttt{@AdminVisible}, następnie dla każdego pola metoda \texttt{handleFieldInClass} obiera strategię jak skonstruować fragment formularza dla konkretnego pola.
\newpage
\begin{small}
	\begin{lstlisting}[language=Java, frame=lines, numberstyle=\tiny, stepnumber=5, caption=Konstrukcja formularza: \texttt{DynamicEntityService.java}\label{listing_form}., firstnumber=1]
	public void constructDynamicEntityForm(String className, Long id,     
	            DynamicForm dynamicForm, 
	            List<RelationMetadata> relationalEntities) {
	    List<Field> fields = getAllFieldsFromClass(className);
	    AbstractEntity entity = dynamicEntityDao
	                            .findAllPolimorficEntityWithId(className, id);
	    for (Field field : fields) {
	        for (Class<?> cls : ExtensionUtil.getSubclassesOf(className)) {
	            handleFieldInClass(cls, field, entity, dynamicForm, 
	            relationalEntities);
	        } 
	    }
	}
	
	private List<Field> getAllFieldsFromClass(String className) {
	    return ExtensionUtil.getPolymorphicFielsdOf(className)
	           .stream()
	           .filter(field -> field.isAnnotationPresent(AdminVisible.class))
	           .sorted(Comparator.comparingInt(field -> 
	                   field.getAnnotation(AdminVisible.class).order()))
	           .collect(Collectors.toList());
	}
	\end{lstlisting} 
\end{small}

Kod źródłowy \ref{listing_handler} z metodą \texttt{handleClassField} przedstawia pracę mechanizmu determinującego strategię konstrukcji pola w formularzu. System rozpoznaje 3 typy pól: kolekcje jeden-do-wielu, kolekcje wiele-do-wielu oraz pola proste (string, bool, int). Dla każdego z nich zaimplementowano osobną metodę, która dodaje do formularza edycji aktualnie obsługiwane pole.
 
\begin{small}
	\begin{lstlisting}[language=Java, frame=lines, numberstyle=\tiny, stepnumber=5, caption=Konstrukcja formularza: \texttt{ClassFieldHandler.java}\label{listing_handler}., firstnumber=1]
	private void handleFieldInClass(Class cls, Field field, 
	             AbstractEntity entity, DynamicForm dynamicForm, 
	             List<RelationMetadata> relationalEntities){
	    Field classField = cls.getDeclaredField(field.getName());
	    classField.setAccessible(true);
	
	    if (Collection.class.isAssignableFrom(classField.getType()) && 
	    classField.getAnnotation(OneToMany.class) != null) {
	        handleOneToManyField(field,entity,relationalEntities);
	    } else if (Collection.class.isAssignableFrom(classField.getType()) 
	               && classField.getAnnotation(ManyToMany.class) != null) {
	        handleManyToManyField(field,entity,relationalEntities);
	    } else {
	        handlePlainField(field,entity,classField,dynamicForm);
	    }
	}
	\end{lstlisting} 
\end{small}

Po zakończeniu budowy formularza kontroler z listingu \ref{listing_dec} skonstruowany formularz oraz dodatkowe informacje do modelu (\texttt{model.addAttribute}). Następnie zwraca nazwę widoku (pliku HTML) z szablonem formularza encyjnego. W przypadku braku znalezienia odpowiedniego uprawnienia, użytkownikowi wyświetlany jest widok z brakiem uprawnień. 