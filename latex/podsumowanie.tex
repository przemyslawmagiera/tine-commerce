\chapter{Podsumowanie}
\thispagestyle{chapterBeginStyle}

Na podstawie niniejszej pracy został zaimplementowany szablonowy system, który pozwala na stworzenie sklepu internetowego opierającego się na gotowych rozwiązaniach. Wraz z platformą przychodzi szereg funkcjonalności, które są kluczowe dla każdego sklepu internetowego. 

Proces implementacji wiązał się z wieloma wyzwaniami. Najtrudniejszym z nich było stworzenie \textbf{elastycznego modelu}\footnote{spełnione wymagania funkcjonalne zdefiniowane w rozdziale \textbf{Analiza problemu} zostały wytłuszczone} oraz \textbf{reużywalnego i rozszerzalnego panalu administracyjnego}. Jest to jedno z ciekawszych rozwiązań, które pozwala na dowolną modyfikację modelu oraz wprowadzanie nowych funkcjonalności. Do systemu \textbf{podłączono zewnętrzny mechanizm indeksujący oraz relacyjną i nierelacyjną bazę danych}. Wraz z platformą użytkownicy mają do dyspozycji \textbf{zaawansowaną wyszukiwarkę Apache Solr}.

Do panelu administracyjnego dodano \textbf{funkcjonalność obsługi uprawnień} w obrębie użytkowników administracyjnych. Stworzono również \textbf{system klasyfikacyjny}, który umożliwia nadawanie dowolnych cech produktowi w zależności od kategorii, z której pochodzi. We frameworku został zaimplementowany także \textbf{pełny proces zakupowy} 

Wszystkie wymagania funkcjonalne określone we wstępie zostały spełnione, jednak biorąc pod uwagę charakter projektu, część z nich wymaga ciągłego udoskonalania. Platformy e-commerce są tworzone latami przez duże zespoły programistów, niemniej jednak projekt ten po dwumiesięcznym okresie implementacji można określić mianem konkretnego \textit{proof of concept}, który jest dobrym fundamentem do stworzenia bardzo dużego i jeszcze bardziej ogólnego rozwiązania. Warta przemyślenia byłby bardziej szczegółowy opis procesu zamówienia oraz, co byłoby bardzo praktyczne, to CMS, czyli system zarządzania treścią w sklepie, aby zapewnić możliwość edycji widoku sklepu z poziomu panelu administracyjnego. Interesującym pomysłem upraszczającym proces wdrożenia byłoby także zrobienie inicjalizera do projektów opartych na opisywanym frameworku, podobnego jak np. na stronie Spring Boot. 

Łącząc wszystkie zaimplementowane funkcjonalności w całość, otrzymamy system, który umożliwi nam szybką implementację sklepu internetowego. Dzięki tej pracy zaoszczędzony czas i pieniądze mogą zostać spożytkowane na lepszy marketing lub design.  
