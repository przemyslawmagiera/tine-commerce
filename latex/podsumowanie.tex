\chapter{Podsumowanie}
\thispagestyle{chapterBeginStyle}

Na podstawie niniejszej pracy został zaimplementowany szablonowy system, który pozwala na stworzenie sklepu internetowego opierającego się na gotowych rozwiązaniach. Wraz z platformą przychodzi szereg funkcjonalności, które są kluczowe dla każdego sklepu internetowego. 

Proces implementacji wiązał się z wieloma wyzwaniami. Najtrudniejszym z nich było stworzenie \textbf{elastycznego modelu}\footnote{spełnione wymagania funkcjonalne zdefiniowane w rozdziale \textbf{Analiza problemu} zostały wytłuszczone} oraz \textbf{reużywalnego i rozszerzalnego panalu administracyjnego}. Z założenia musiał on zapewnić obsługę dodawania, usuwania i edycji dowolnej encji, którą zdefiniujemy w systemie. Największym wyzwaniem okazała się generyczna edycja encji oraz jej relacji. Jest to jedno z ciekawszych rozwiązań, które pozwala na dowolną modyfikację modelu oraz wprowadzanie nowych funkcjonalności bardzo niskim kosztem, gdyż platforma zapewnia mechanizm obsługi dowolnej funkcjonalności w sklepie, nawet gdy nie została przewidziana bezpośrednio przez twórcę frameworku i musi zostać dodana. Wymagało to napisania wielu dynamicznych kwerend bazodanowych oraz obsługi obiektów przez refleksję w języku Java.

Kolejnymi spełnionymi wymaganiami są \textbf{podłączony zewnętrzny mechanizm indeksujący, relacyjna i nierelacyjna baza danych} oraz \textbf{wydajna wyszukiwarka}. Wraz z platformą użytkownicy mają do dyspozycji zaawansowaną wyszukiwarkę Apache Solr. Jej integracja z frameworkiem wiązała się zaprojektowaniem i zaimplementowaniem serwisów, których zadaniem jest eksport produktów i wszystkich jego cech z bazy relacyjnej do bazy noSQL Solra. W tym miejscu znowu należało przemyśleć kwestię generyczności rozwiązania i część z metod oprzeć na refleksji i elastycznym modelu w związku z zachowaniem możliwości nadpisania i zmiany cech produktu. Warto wspomnieć również, że dzięki indeksacji i płaskiej strukturze danych wyszukiwarka jest bardzo wydajna, do tego jest umieszczona na osobnym serwerze, co umożliwia jej skalowalność. 

Do panelu administracyjnego dodano \textbf{funkcjonalność obsługi uprawnień} w obrębie użytkowników administracyjnych. Opierają się one na strukturze drzewiastej, co zapewnia możliwość dziedziczenia ich między sobą i konfigurowania wedle życzenia klienta. Na podobnej strukturze oparto \textbf{system klasyfikacyjny}, który umożliwia nadawanie dowolnych cech produktowi w zależności od kategorii, z której pochodzi.

We frameworku został zaimplementowany \textbf{pełny proces zakupowy} wraz z przykładowym sklepem, gdzie możliwe jest jego wypróbowanie. Klient może wyszukać produkt, wyświetlić jego szczegóły, zalogować się, dodać go do koszyka i potwierdzić zamówienie. Proces ten został oczywiście zintegrowany z panelem administracyjnym, gdzie jest możliwość zarządzania zamówieniami oraz kupującymi.

Wszystkie wymagania funkcjonalne określone we wstępie zostały spełnione, jednak biorąc pod uwagę charakter projektu, część z nich wymaga ciągłego udoskonalania. Platformy e-commerce są tworzone latami przez duże zespoły programistów, niemniej jednak projekt ten po dwumiesięcznym okresie implementacji można określić mianem konkretnego \textit{proof of concept}, który jest dobrym fundamentem do stworzenia bardzo dużego i jeszcze bardziej ogólnego rozwiązania. Warta przemyślenia byłby bardziej szczegółowy opis procesu zamówienia oraz, co byłoby bardzo praktyczne, to CMS, czyli system zarządzania treścią na sklepie, aby zapewnić możliwość edycji widoku sklepu z poziomu panelu administracyjnego. Interesującym pomysłem upraszczającym proces wdrożenia byłoby także zrobienie inicjalizera do projektów opartych na opisywanym frameworku, podobnego jak np. na stronie Spring Boot. 

Łącząc wszystkie zaimplementowane funkcjonalności w całość, otrzymamy system, który umożliwi nam szybką implementację sklepu internetowego. Framework dostarczy nam wydajną wyszukiwarkę, katalog produktowy i zaawansowany panel administracyjny, czyli fundament, który podczas zwykłego dewelopmentu zwykle zabiera najwięcej czasu. Dzięki tej pracy zaoszczędzony czas i pieniądze mogą zostać spożytkowane na lepszy marketing lub design. Taki był właśnie cel niniejszej pracy.  




